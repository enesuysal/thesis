%%%%%%%%%%%%%%%%%%%%%%%%%%%%%%%%%%%%%%%%%%%%%%%%%%%%%%%%%%%%%%%%%%%%%%%%
%                                                                      %
%     File: Thesis_Abstract.tex                                        %
%     Tex Master: Thesis.tex                                           %
%                                                                      %
%     Author: Andre C. Marta                                           %
%     Last modified :  2 Jul 2015                                      %
%                                                                      %
%%%%%%%%%%%%%%%%%%%%%%%%%%%%%%%%%%%%%%%%%%%%%%%%%%%%%%%%%%%%%%%%%%%%%%%%

\section*{Abstract}

% Add entry in the table of contents as section
\addcontentsline{toc}{section}{Abstract}

Em ambiente cloud, as soluções de programação distribuída, contemplando interoperabilidade entre sistemas heterogéneos,
são essencialmente baseadas em tecnologia SOA ou REST, com base em HTTP, XML e JSON. O problema é que estas tecnologias
foram desenvolvidas no contexto Web, em que o principal objetivo é a integração de sistemas existentes e não a programação
distribuída, com eficiência e desempenho como preocupações de topo. Como conciliar estas duas visões, em ambiente global,
sobre cloud computing?



SOA e Web Services são tecnologias pesadas e complexas. A popularidade do REST deve-se mais à sua simplicidade do que ao seu
mapeamento no paradigma dos serviços. No entanto, qualquer delas é pouco eficiente, baseando-se no HTTP e em linguagens de
descrição de dados baseadas em texto (XML e JSON). O resultado é que a descrição de Web Services (WSDL) e linguagens de
programação de alto nível, como BPEL, ficam ineficientes e complexas. A versão binária do XML não é mais do que uma compressão
de dados apenas para efeito de transmissão (comprimida na emissão e descomprimida na receção). A linguagem a usar neste
trabalho usa formato binário de dados de forma nativa, não necessita de schemas e o ambiente de execução usa Web Sockets,
muito mais eficientes do que HTTP, embora mantendo alguma compatibilidade.

Usar uma linguagem, já existente, desenvolvida para programação distribuída em ambiente de cloud computing, para desenvolver
aplicações distribuídas, comparando esta solução com as soluções mais clássicas baseadas em SOA e REST.
Resultado esperado: Um protótipo de aplicação distribuída, de nível de negócio, sobre a linguagem referida e o seu ambiente
de execução (já existentes). Estudo comparativo com as soluções correspondentes com tecnologias baseadas em SOA (web Services)
e REST.

\vfill

\textbf{\Large Keywords:} SOA, WSDL, REST, XML, JSON, HTTP, Web Sockets
