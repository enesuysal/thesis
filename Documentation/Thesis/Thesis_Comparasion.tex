%%%%%%%%%%%%%%%%%%%%%%%%%%%%%%%%%%%%%%%%%%%%%%%%%%%%%%%%%%%%%%%%%%%%%%%%
%                                                                      %
%     File: Thesis_Results.tex                                         %
%     Tex Master: Thesis.tex                                           %
%                                                                      %
%     Author: Andre C. Marta                                           %
%     Last modified :  2 Jul 2015                                      %
%                                                                      %
%%%%%%%%%%%%%%%%%%%%%%%%%%%%%%%%%%%%%%%%%%%%%%%%%%%%%%%%%%%%%%%%%%%%%%%%



\chapter{Comparison with existing technologies}
\label{chapter:comparison}


\section{Overview}
\label{section:overview}

Current integration technologies for distributed systems in Cloud environment are generally supported on
textual data description languages (XML and JSON) and HTTP protocol.
These technologies especially designed for human-level interaction and that creates integration problems.
In this section we will do comparison qualitatively and quantitevely with these current technologies and our new approach.

%%%%%%%%%%%%%%%%%%%%%%%%%%%%%%%%%%%%%%%%%%%%%%%%%%%%%%%%%%%%%%%%%%%%%%%%

\section{Qualitatively Comparison}
\subsection{Textual or binary representation}
\label{section:textualbinary}
Current integration technologies for distributed systems in Cloud environment are generally supported on
textual data description languages (XML and JSON) and HTTP protocol.
These technologies especially designed for human-level interaction and that creates integration problems.
In this section we will do comparison with these current technologies and our new approach.\\

SOA is usually implemented by Web Services with WSDL, which is a set of conventions on XML usage to describe
services at the interface level and SOAP as a message protocol, which is again based on XML.\\

Many developers found SOAP cumbersome and hard to use. For example, working with SOAP in JavaScript means writing
a ton of code to perform extremely simple tasks because you must create the required XML structure absolutely
every time. one perceived disadvantage is the use of XML because of the verboseness of it and the time it takes
to parse.\\


REST also requires that data types, which are usually, called media types and standardized or previously agreed,
when they are application specific.
REST doesn’t have to use XML to provide the response. You can find REST-based Web services that output
the data in Command Separated Value (CSV), JavaScript Object Notation (JSON) and Really Simple Syndication (RSS).
The point is that you can obtain the output you need in a form that’s easy to parse within the language you need
for your application. While this may seem like it adds complexity to REST because you need to handle multiple
formats, JSON usually is a better fit for data and parses much faster. REST allows better support for browser
clients due to its support for JSON.

SOA and REST use textual representation but textual representation brings parsing expenses and poor support
for binary data. Instead of using textual representation, the binary representation provides native support for
binary data, has a smaller length and is faster to parse.

Our approach is designed to work with binary representation. The binary representation uses a modified version
of the TLV format (Tag, Length and Value) used by ASN.1 \citep{asn1:opt}. This not only supports the direct integration of
binary information but also facilitates parsing, since each resource, primitive or structured.

%%%%%%%%%%%%%%%%%%%%%%%%%%%%%%%%%%%%%%%%%%%%%%%%%%%%%%%%%%%%%%%%%%%%%%%%
\subsection{Message protocol}
\label{section:mprotocol}

SOAP is language, platform, and transport independent SOAP can use almost any transport to send the request,
using everything from the afore mentioned to SMTP (Simple Mail Transfer Protocol) and even
JMS (Java Messaging Service), but REST requires use of HTTP/HTTPS. \\
Our approach does not depend on any particular transport protocol, relying only on message delivery.
Any existing server can be used, based on HTTP, WebSockets or any other protocol. In fact, several servers can
be used simultaneously, receiving messages that are handed over to the message handlers that are able to process
them. \\
New protocols, such as WebSockets, reduce some of the problems. Web Sockets,
now part of the HTML5 world, removes this restriction, adds binary support and increases performance.


XML is verbose and complex, has limited support for binary formats, is inefficient in computer terms due to
parsing and exhibits symmetric interoperability, based on both sender and receiver using the same schema,
which constitutes a relevant coupling problem.

\\


%%%%%%%%%%%%%%%%%%%%%%%%%%%%%%%%%%%%%%%%%%%%%%%%%%%%%%%%%%%%%%%%%%%%%%%%
\subsection{The interoperability problem}
\label{section:interoperabilityProblem}

In both SOA and REST, interoperability is achieved by using common data types (usually structured),
either by sharing a schema (i.e., WSDL files) or by using a previously agreed data type
(typical in RESTful applications). There is usually no support for partial interoperability and
polymorphism in distributed systems.\\

The basis of data interoperability with XML and JSON is schema sharing(at runtime or with previously standardized
or agreed upon internet media types). Both the producer and consumer (reader) of a document should use the same
schema, to ensure that any document produced (with that schema) can be read by the consumer. This means that
both producer and consumer will be coupled by this schema. Schemas must be shared between interacting Web Services,
 establishing coupling for all the possible values satisfying each schema, even if they are not actually used.
 In this case, a reference to a schema acts like its name.\\


REST also requires that data types (usually called media types) must have been previously agreed, either
standardized or application specific.\\
Searching for an interoperable Web Service is usually done by schema matching with similarity algorithms \citep{expertsystems:x1}
and ontology matching and mapping \citep{Ontology:matching}. This does not ensure interoperability and manual adaptations are usually
inevitable.
\\
In our solution, we propose to use partial interoperability, based on the concepts of compliance and conformance.
It introduces a different
perspective, stronger than similarity but weaker than commonality (sharing). The trick is to allow partial
interoperability, by considering only the intersection between what the consumer needs and what the provider offers.
It allow us for increased interoperability,
adaptability and changeability, without the need to have resource types
necessarily shared or previously agreed.

Building interoperability on compliance and conformance avoids the problem of having to define schemas as
separate documents and to agree upon them beforehand. As long as compliance and conformance hold, any two
resources can interoperate, even if they were designed unawares to each other.

\section{Quantatively Comparison}
In this section we present some results obtained with implemented solution and current solutions. We tested our
tool in Microsoft Azure Cloud Service.

In Figure 15 we have a graph where the execution times of each test case are visible. For each test case the time of the Front-end implementation represents the PinaVM runtime.

Figure 15 - Execution times (milliseconds).
