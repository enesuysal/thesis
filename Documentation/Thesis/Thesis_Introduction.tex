%%%%%%%%%%%%%%%%%%%%%%%%%%%%%%%%%%%%%%%%%%%%%%%%%%%%%%%%%%%%%%%%%%%%%%%%
%                                                                      %
%     File: Thesis_Introduction.tex                                    %
%     Tex Master: Thesis.tex                                           %
%                                                                      %
%     Author: Andre C. Marta                                           %
%     Last modified :  2 Jul 2015                                      %
%                                                                      %
%%%%%%%%%%%%%%%%%%%%%%%%%%%%%%%%%%%%%%%%%%%%%%%%%%%%%%%%%%%%%%%%%%%%%%%%

\chapter{Introduction}
\label{chapter:introduction}

Distributed applications are a necessity in most central application sectors of the modern information society,
including e-commerce, e-bank, e-learn, e-health, telecommunication and transportation. This results from a huge
growth of the role that the Internet plays in business, administration and our everyday activities.
The fundamental problem is the programming of distributed applications with all the basic interoperability problems
involving distributed platforms and heterogeneous components.\\

On the other way, Clouds create a new challenge for distribution. They bring new opportunities such as scalability,
dynamic instantiation, location independence, application management and multi-tenancy (several users using the same
application independently). However, they also create distributed platforms and that’s why they need to be able to
support distributed applications as easily as possible.\\

Taking as an example the your Android mobile phone, let's say you have an application in your phone that informs you with current
weathercast of your city. That application is most probably written in Java since your phone is using Android system and gets
current weathercast over Internet, so basically that application asks query to weathercast provider over Internet and gets
response data then parse that data and display to you. During this request and response both your mobile phone and weathercast
provider must understand each other even they don’t use the same language because the provider could be working in a Cloud
provider and written in C\# language. This interoperability (how to interconnect different programs written in different
languages, running in different platforms) issue can be solved with current integration technologies such as SOA or REST and both
platforms can communicate between each other. You can get last weathercast information from your Android phone or your IPhone
using same weathercast provider.\\

The traditional integration technologies, based on either SOA or REST, is that they use the document concept as the foundation,
with a data description language as the representation format and schema sharing as the interoperability mechanism (both sides
use the same schema such XML Schema). Other factors, such as a connectionless protocol (HTTP) and the lack of native support for
binary data because the solutions were also based on text (XML) and contextual information, are limiting for many applications.\\

As a solution to the problems mentioned above, in our solution we use binary directly, instead of text and compliance and
conformance instead of sharing the same schema. The main task of this solution is to show that there can be an alternative
solution to XML-based technologies, for document sharing or service invocation between two completely different systems.\\

The objective of dissertation is aimed to show an alternative solution to Web Services and REST.
Implementing a distributed application with a client programmed in one language and a remote service programmed in possibly
another language.\\

The main goal of this dissertation is more than just an implementation and the value of this dissertation lie in the
demonstration of conclusions, with respect to Web Services and REST. Showing results that if it is a better solution for
application interoperability and also if it is easier to implement. We also assess performance, with comparison current solutions.\\

This remainder of this dissertation is organized as follows:

\begin{itemize}
\item State of Art - Chapter 2 details some aspects of existing solutions for distributed systems for cloud environment, namely SOA and REST. it also details
about new tools that we will use, namely Web Sockets. 
\item Interoperability - Chapter 3 starts describing interoperability and explaining different perspective from classical
solutions to our new solution to overcome interoperability problem.
\item Architecture of the solution - Chapter 4 provides some insight on how our system can be used and gives a general overview of
how and why it works.
\item Implementation - Chapter 5 goes more in depth on the inner workings of our system than the previous chapter and presents one implementation
for our system.
\item Comparison with existing technologies - Chapter 6 presents the benchmarks used to evaluate our system with current solutions and
the results obtained.
\item Conclusions - Chapter 7 summarizes the work described in this dissertation, the results achieved, and what are its main contributions.
It also presents some possible future improvements to our proposed solution.
\end{itemize}
