%%%%%%%%%%%%%%%%%%%%%%%%%%%%%%%%%%%%%%%%%%%%%%%%%%%%%%%%%%%%%%%%%%%%%%%%
%                                                                      %
%     File: Thesis_Abstract.tex                                        %
%     Tex Master: Thesis.tex                                           %
%                                                                      %
%     Author: Andre C. Marta                                           %
%     Last modified :  2 Jul 2015                                      %
%                                                                      %
%%%%%%%%%%%%%%%%%%%%%%%%%%%%%%%%%%%%%%%%%%%%%%%%%%%%%%%%%%%%%%%%%%%%%%%%

\section*{Abstract}

% Add entry in the table of contents as section
\addcontentsline{toc}{section}{Abstract}

In Cloud environment, programming distributed solutions, covering interoperability between heterogeneous systems,
they are essentially based on SOA or REST technology, based on HTTP, XML and JSON. The problem is that these technologies
were developed in the web context where the main objective is the integration of existing systems and not programming
distributed with efficiency and performance as top concerns. How to compromise these two views in the global environment,
on cloud computing?\\

SOA and Web services are heavy and complex technologies. The popularity of REST is due more to its simplicity than to its
mapping the paradigm of services. However, any of them is not very efficient, based on HTTP and languages
text-based data description (XML and JSON). The result is that the description of Web Services (WSDL) and languages
high-level programming are inefficient and complex. The binary version of XML is no more than a compression
data only for transmission effect (compressed and decompressed on the issue at the reception). A solution used in this
work uses binary data format natively, does not require schema and the runtime environment uses Web Sockets,
much more efficient than HTTP, while maintaining some compatibility.\\

New solution is developed for distributed programming in cloud computing environment to develop
distributed applications, using  asymmetric interoperability, based on compliance and conformance and comparing this solution with the more classical solutions based on SOA and REST. The aim of this work explains this solution and its environment execution. Comparison with the corresponding solutions with technologies based on SOA (Web Services)
and REST.

\vfill

\textbf{\Large Keywords:} SOA, WSDL, REST, XML, JSON, HTTP, Web Sockets
