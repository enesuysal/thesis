%%%%%%%%%%%%%%%%%%%%%%%%%%%%%%%%%%%%%%%%%%%%%%%%%%%%%%%%%%%%%%%%%%%%%%%%
%                                                                      %
%     File: Thesis_Abstract.tex                                        %
%     Tex Master: Thesis.tex                                           %
%                                                                      %
%     Author: Andre C. Marta                                           %
%     Last modified :  2 Jul 2015                                      %
%                                                                      %
%%%%%%%%%%%%%%%%%%%%%%%%%%%%%%%%%%%%%%%%%%%%%%%%%%%%%%%%%%%%%%%%%%%%%%%%

\section*{Abstract}

% Add entry in the table of contents as section
\addcontentsline{toc}{section}{Abstract}

In Cloud environment, distributed programming solutions which cover interoperability between heterogeneous systems are essentially based on SOA or REST technology that uses HTTP, XML and JSON. The problem is that these technologies
were developed in the web context where the main objective is the integration of existing systems and not programming
distributed with efficiency and performance as top concerns. How to compromise these two views in the global environment,
on cloud computing?\\

SOA and Web services are heavy and complex technologies. The popularity of REST is more due to its simplicity than due to its
mapping the paradigm of services. However, both of them is not very efficient because they are based on HTTP or
text-based data description languages (XML and JSON). The result is that the description of Web Services (WSDL) and languages
high-level programming are inefficient and complex. The binary version of XML is no more than a compression
data only for transmission effect (compressed and decompressed on the issue at the reception). A solution used in this work uses binary data format natively and does not require schema and uses Web Sockets as the runtime environment and these Web Sockets are much more efficient than HTTP, while maintaining some compatibility.\\

New solution is developed for distributed programming in cloud computing environment to develop distributed applications. While this solution compared with classical solution which is based on SOA and REST, this solution uses asymmetric interoperability which is based on compliance and conformance. The aim of this work is to explain this solution and its execution environment and also to compare it with the corresponding solution technologies which are based on SOA (Web Services) and REST.


\vfill

\textbf{\Large Keywords:} SOA, WSDL, REST, XML, JSON, HTTP, Web Sockets
