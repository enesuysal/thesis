%%%%%%%%%%%%%%%%%%%%%%%%%%%%%%%%%%%%%%%%%%%%%%%%%%%%%%%%%%%%%%%%%%%%%%%%
%                                                                      %
%     File: Thesis_State_of_the_art.tex                                %
%     Tex Master: Thesis.tex                                           %
%                                                                      %
%                                                                      %
%%%%%%%%%%%%%%%%%%%%%%%%%%%%%%%%%%%%%%%%%%%%%%%%%%%%%%%%%%%%%%%%%%%%%%%%

\chapter{State-of-the-art}
\label{chapter:stateofart}


%%%%%%%%%%%%%%%%%%%%%%%%%%%%%%%%%%%%%%%%%%%%%%%%%%%%%%%%%%%%%%%%%%%%%%%%
\section{Overview}
\label{section:overview}

(description of existing technologies and tools that you will use)
Existing Technologies

- SOA (Youtube)

- REST (YouTube)

Tools
Tomcat
.Net Framework 4.5

Language (C# - Java)
Deployment -
Azure Web Services Cloud Technoloies
Insert your chapter material here...

%%%%%%%%%%%%%%%%%%%%%%%%%%%%%%%%%%%%%%%%%%%%%%%%%%%%%%%%%%%%%%%%%%%%%%%%

\section{Web Services}
\label{section:webservices}

Web Services are exposed to the Internet for programmatic access. They are online APIs that you can call from your code.\\
When you want to call any API that written by someone else to your java code, you basically add jar or classes to
your class path and executions are done inside of machine or single environment. In the case of web services however
you have different pieces of code deployed over different machines and they call methods of each other over the network.
For example, you must have seen that different apps or games, which can post to your Facebook wall even these games,
not designed by Facebook. So you ask that how they can do that or how they can post to a wall of completely different
system or application. Basically they do this by calling online APIs. Companies like Facebook or Twitter publish web
services that let other developers call them from their code, so other application developers can actually write
code to consume these services and they can post things on Facebook or Twitter. They can read or access data from
Facebook or Twitter using the APIs of the web services that Facebook or Twitter has provided.\\
Web services are similar to web pages. For example Twitter has web side URL as “www.twitter.com“ when you access
this URL on your browser you get an HTML response that let you read and write tweets. They have HTML elements for
data and also CSS files for styling, this is because web pages that you see is made human conception.
They know that there is actually human is behind of browser on a laptop or devices who was reading these tweets
so they want to make sure about its format properly, so it is easy to access and read.
Twitter has also other URL as “api.twitter.com“ that does a lot of same things as “www.twitter.com“ does,
but it behaves a bit differently for instance this API gives you response which doesn’t have HTML or CSS code.
It contains data but it is xml or json format and there are specific URLs for different operations this is what
the developers can use from their code to read or write to twitter, so this data is actually very easy for
parsing and converting then using in their objects and their code for developers. In this case there is no need HTML
and CSS files.\\
There are primarily two different types of web services. One of these types called as SOA web service and another type
called as REST web service. SOA is older of these two and REST is newer entry to web services world, but both of them
are used popular. Next chapter will be focusing on SOA and REST web services.

%%%%%%%%%%%%%%%%%%%%%%%%%%%%%%%%%%%%%%%%%%%%%%%%%%%%%%%%%%%%%%%%%%%%%%%%
\section{SOA Web Services}
\label{section:soa}
-WSDL
-

%%%%%%%%%%%%%%%%%%%%%%%%%%%%%%%%%%%%%%%%%%%%%%%%%%%%%%%%%%%%%%%%%%%%%%%%
\section{Theoretical Model 1}
\label{section:theory1}

The research should be supported with a comprehensive list of references.
These should appear whenever necessary, in the limit, from the first to the last chapter.

A reference can be cited in any of the following ways:
%
\begin{itemize}
  \item Citation mode \#1 - \quad \cite{jameson:adjointns}
  \item Citation mode \#2 - \quad \citet{jameson:adjointns}
  \item Citation mode \#3 - \quad \citep{jameson:adjointns}
  \item Citation mode \#4 - \quad \citet*{jameson:adjointns}
  \item Citation mode \#5 - \quad \citep*{jameson:adjointns}
  \item Citation mode \#6 - \quad \citealt{jameson:adjointns}
  \item Citation mode \#7 - \quad \citealp{jameson:adjointns}
  \item Citation mode \#8 - \quad \citeauthor{jameson:adjointns}
  \item Citation mode \#9 - \quad \citeyear{jameson:adjointns}
  \item Citation mode \#10 - \quad \citeyearpar{jameson:adjointns}
\end{itemize}
%
Several citations can be made simultaneously as \citep{nocedal:opt,marta:ijcfd}. \\

This is often the default bibliography style adopted (numbers following the citation order), according to the options:\\
{\tt \textbackslash usepackage\{natbib\}} in file {\tt Thesis\_Preamble.tex},\\
{\tt \textbackslash bibliographystyle\{abbrvnat\}} in file {\tt Thesis.tex}.\\
%
Notice however that this style can be changed from numerical citation order to authors' last name with the options: \\
{\tt \textbackslash usepackage[numbers]\{natbib\}} in file {\tt Thesis\_Preamble.tex},\\
{\tt \textbackslash bibliographystyle\{abbrvunsrtnat\}} in file {\tt Thesis.tex}.


%%%%%%%%%%%%%%%%%%%%%%%%%%%%%%%%%%%%%%%%%%%%%%%%%%%%%%%%%%%%%%%%%%%%%%%%
\section{Theoretical Model 2}
\label{section:theory2}

Other models...
