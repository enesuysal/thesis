%%%%%%%%%%%%%%%%%%%%%%%%%%%%%%%%%%%%%%%%%%%%%%%%%%%%%%%%%%%%%%%%%%%%%%%%
%                                                                      %
%     File: Thesis_State_of_the_art.tex                                %
%     Tex Master: Thesis.tex                                           %
%                                                                      %
%                                                                      %
%%%%%%%%%%%%%%%%%%%%%%%%%%%%%%%%%%%%%%%%%%%%%%%%%%%%%%%%%%%%%%%%%%%%%%%%

\chapter{State-of-the-art}
\label{chapter:stateofart}


%%%%%%%%%%%%%%%%%%%%%%%%%%%%%%%%%%%%%%%%%%%%%%%%%%%%%%%%%%%%%%%%%%%%%%%%
\section{Overview}
\label{section:overview}

(description of existing technologies and tools that you will use)
Existing Technologies

- SOA (Youtube)

- REST (YouTube)

Tools
Tomcat
.Net Framework 4.5

Language (C# - Java)
Deployment -
Azure Web Services Cloud Technoloies
Insert your chapter material here...

%%%%%%%%%%%%%%%%%%%%%%%%%%%%%%%%%%%%%%%%%%%%%%%%%%%%%%%%%%%%%%%%%%%%%%%%

\section{Web Services}
\label{section:webservices}

Web Services are exposed to the Internet for programmatic access. They are online APIs that you can call from your code.\\
When you want to call any API that written by someone else to your java code, you basically add jar or classes to
your class path and executions are done inside of machine or single environment. In the case of web services however
you have different pieces of code deployed over different machines and they call methods of each other over the network.
For example, you must have seen that different apps or games, which can post to your Facebook wall even these games,
not designed by Facebook. So you ask that how they can do that or how they can post to a wall of completely different
system or application. Basically they do this by calling online APIs. Companies like Facebook or Twitter publish web
services that let other developers call them from their code, so other application developers can actually write
code to consume these services and they can post things on Facebook or Twitter. They can read or access data from
Facebook or Twitter using the APIs of the web services that Facebook or Twitter has provided.\\
Web services are similar to web pages. For example Twitter has web side URL as “www.twitter.com“ when you access
this URL on your browser you get an HTML response that let you read and write tweets. They have HTML elements for
data and also CSS files for styling, this is because web pages that you see is made human conception.
They know that there is actually human is behind of browser on a laptop or devices who was reading these tweets
so they want to make sure about its format properly, so it is easy to access and read.
Twitter has also other URL as “api.twitter.com“ that does a lot of same things as “www.twitter.com“ does,
but it behaves a bit differently for instance this API gives you response which doesn’t have HTML or CSS code.
It contains data but it is xml or json format and there are specific URLs for different operations this is what
the developers can use from their code to read or write to twitter, so this data is actually very easy for
parsing and converting then using in their objects and their code for developers. In this case there is no need HTML
and CSS files.\\
There are primarily two different types of web services. One of these types called as SOA web service and another type
called as REST web service. SOA is older of these two and REST is newer entry to web services world, but both of them
are used popular. Next chapter will be focusing on SOA and REST web services.

%%%%%%%%%%%%%%%%%%%%%%%%%%%%%%%%%%%%%%%%%%%%%%%%%%%%%%%%%%%%%%%%%%%%%%%%
\section{SOA Web Services}
\label{section:soa}
DEFINITON

xXXX
https://books.google.fr/books/about/Understanding_Web_Services.html?id=SHSBri-rMyQC&redir_esc=y

EXAMPLE

Let’s start an example with java application. Let’s say we have implementation class and I want to share this implementation
class with other developer projects. How I would share this with a consumer class. The best way to share this implementation
class would be to contract it with an interface and other consumers would consume this class through this interface.
They would call implementation through interface so they get contract and they get the methods, arguments, written types
through interface. They actually call the methods of implementation class, so how this works in case of web service, let’s
say I have web service implementation and I want to share details of this web service to consumers. Is it works with an
interface? Probably it will not work because as we discussed before you don’t know what technology is consuming it. It might
be C# application or C++ application, so if you have a java web service you might want to give some kind of information that
its consumer respects to that technology and that can actually consume. Let’s say consumer is .Net and if I give this .Net
application a java interface then most probably it will not work because they are different technologies, so the technology
that I am going to share with web service consumer has to be technology independent. It should be something that any
application or any technology can understand so creators of SOA web service talked about that problem and what to give format
understandable by all technology with all consumers and decided with XML. So what you do is in case of web service, you
actually share that contract as an XML document. This XML document is actually called as WSDL.\\

[GRAPH]  -> interface similar to video\\

WSDL document contains the contract to web service and so that’s are the things you have to do when you create the web
service and you share WSDL document of that web service to the consumers, so this is not something you would have to do
it manually. You would do it manually but there are tools which generate WSDL for the web service but it is something that
you need to share this WSDL to consumers and it is a XML document, so it respective whatever application because applications
such as .Net, C++ or Java can all parse this XML and get to know about service information and typically the content of
this WSDL is kind of similar to an interface content. It has operations, arguments and types  to return that consumer
applications will have idea what to call and how to call.\\

[GRAPH] WSDL similar to video\\

The new question is that how this exchange happens, how you actually send this information, let’s say you have a method in
your java application and input argument is a string so you have a java string with you and you need to send to web service
and let’s say output return type is a list, so how you get this information because that could be .Net application and string
in java obviously different from string in C#. How do you exchange this between client app and web service?  When you exchange
information input argument or return type you need to exchange it in the format that all different technologies can understand
what you are passing and it should be able to send return type back in language that all these technologies can understand.
Again this format is XML. When you are sending any information across the network from a client to the web services and
return type back to the client, the data has to be in XML format. You are not really sending java string or a list. So it
has to be language natural format which is XML. There is specification about how you need to send all these different input
type and output argument basically any type needs to be send specific XML format.\\

It is a protocol that is a way in which both sender and receiver and this XML is called SOAP (Simple Object Access Protocol).
It is a way in which these different technologies can access objects can access data it supposedly simple so that a part
of the name is called simple object access protocol. So with this protocol all different technologies written in different
languages can kind of understand what they all taking about.\\

[GRAPH] SOAP\\

Now you know what is the mechanism you know what need to be send and you know how to send which is using SOAP protocol
but who does conversion? So for example you have your string object or complex object, so how do you convert it from java
object to a soap message? The conversion is actually done with intermedia class so this class takes care of converting all
your objects into a SOAP message. The whole method calls itself is actually done by SEI. The SEI access interface to your
web service endpoint so you have an interface at your client app to the service endpoint which translate all web service
call to a SOAP message and then it makes sure that the other things is able to understand this message. So we don’t have to
write this class and all the conversion ourselves. We can have it automatically generated for us. When you are making a web
service call you don’t worry about where the web service is. When you need to call, all you need to do is have this endpoint
interface and good thing about this service endpoint that you can actually have an interface that specific to what you are
developing. When you have a java application you will have a specific SEI for java application and it knows to convert java
objects to SOAP message. Let’s say your .Net application calling the same web service so you will have SEI for .NET that know
to convert .Net objects to SOAP message.\\

GRAPH SEI\\


%%%%%%%%%%%%%%%%%%%%%%%%%%%%%%%%%%%%%%%%%%%%%%%%%%%%%%%%%%%%%%%%%%%%%%%%
\section{Theoretical Model 1}
\label{section:theory1}

The research should be supported with a comprehensive list of references.
These should appear whenever necessary, in the limit, from the first to the last chapter.

A reference can be cited in any of the following ways:
%
\begin{itemize}
  \item Citation mode \#1 - \quad \cite{jameson:adjointns}
  \item Citation mode \#2 - \quad \citet{jameson:adjointns}
  \item Citation mode \#3 - \quad \citep{jameson:adjointns}
  \item Citation mode \#4 - \quad \citet*{jameson:adjointns}
  \item Citation mode \#5 - \quad \citep*{jameson:adjointns}
  \item Citation mode \#6 - \quad \citealt{jameson:adjointns}
  \item Citation mode \#7 - \quad \citealp{jameson:adjointns}
  \item Citation mode \#8 - \quad \citeauthor{jameson:adjointns}
  \item Citation mode \#9 - \quad \citeyear{jameson:adjointns}
  \item Citation mode \#10 - \quad \citeyearpar{jameson:adjointns}
\end{itemize}
%
Several citations can be made simultaneously as \citep{nocedal:opt,marta:ijcfd}. \\

This is often the default bibliography style adopted (numbers following the citation order), according to the options:\\
{\tt \textbackslash usepackage\{natbib\}} in file {\tt Thesis\_Preamble.tex},\\
{\tt \textbackslash bibliographystyle\{abbrvnat\}} in file {\tt Thesis.tex}.\\
%
Notice however that this style can be changed from numerical citation order to authors' last name with the options: \\
{\tt \textbackslash usepackage[numbers]\{natbib\}} in file {\tt Thesis\_Preamble.tex},\\
{\tt \textbackslash bibliographystyle\{abbrvunsrtnat\}} in file {\tt Thesis.tex}.


%%%%%%%%%%%%%%%%%%%%%%%%%%%%%%%%%%%%%%%%%%%%%%%%%%%%%%%%%%%%%%%%%%%%%%%%
\section{Theoretical Model 2}
\label{section:theory2}

Other models...
